\documentclass[a4paper]{article}
\usepackage[italian]{babel}

\title{Esercitazione Fisica\\ Moto Rettiline Uniforme (MRU)}
\author{Enrico Martini}


\begin{document}

\maketitle
\thispagestyle{empty}

Esercizi:

\begin{enumerate}
    \item Un automobilista guida per $35.0$ minuti, ad una velocità
    di $85.0$ km/h, in direzione nord. Si ferma per
    $15.0$ minuti e riprende il viaggio per $2.00$ h, sempre in
    direzione nord, alla velocità di $130$ km/h. Detenninare
    (a) la distanza totale percorsa, (b) la velocità
    media nel percorso.
    \item Un'automobile attraversa un semaforo alla velocità di $80$ km/h. Nello stesso istante, uno scooter che si trova $0.9$ km più avanti, mantiene una velocità di 28 km/h. Determinare (a) quanto tempo ci impiega l'automobile a raggiungere lo scooter, (b) a che distanza dal semaforo si trovano i due veicoli quando avviene il sorpasso.
    \item Due auto gareggiano su $45$ giri di un circuito lungo $8$ km. La prima macchina viaggia a una velocità media di $243$ km/h, la seconda di $232$ km/h. Determinare (a) chi vince tra le due auto, (b) la differenza di tempo tra l'arrivo della prima auto e l'arrivo della seconda auto, (c) al momento dell'arrivo della prima auto quanto dista la seconda.
\end{enumerate}
\vspace{2cm}
\textit{Tempo: 1 ora.}
\\
\textit{Ogni problema dovrà avere la struttura:}
\begin{itemize}
    \item \textit{Dati ricavati dal testo del problema}
    \item \textit{Dati richiesti dal problema}
    \item \textit{Svolgimento}
    \item \textit{Soluzione}
\end{itemize}

\end{document}